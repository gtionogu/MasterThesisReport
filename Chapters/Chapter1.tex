% Chapter 1

\chapter{Introduction} % Main chapter title

\label{Introduction} % For referencing the chapter elsewhere, use \ref{Chapter1} 

%----------------------------------------------------------------------------------------

% Define some commands to keep the formatting separated from the content 
\newcommand{\keyword}[1]{\textbf{#1}}
\newcommand{\tabhead}[1]{\textbf{#1}}
\newcommand{\code}[1]{\texttt{#1}}
\newcommand{\file}[1]{\texttt{\bfseries#1}}
\newcommand{\option}[1]{\texttt{\itshape#1}}

%----------------------------------------------------------------------------------------
\section{Research context}
Vulnerabilties keep growing $\rightarrow$ making botnets even easier to spread, which means cheaper and more powerful.
%https://www.welivesecurity.com/2018/02/05/vulnerabilities-reached-historic-peak-2017/

[source]\\
%\href{http://www.wikicfp.com/cfp/servlet/event.showcfp?eventid=75599&copyownerid=38142}
%\href{https://www.csoonline.com/article/3240364/hacking/what-is-a-botnet-and-why-they-arent-going-away-anytime-soon.html}
%\href{https://researchcenter.paloaltonetworks.com/2018/07/unit42-finds-new-mirai-gafgyt-iotlinux-botnet-campaigns/}
%\href{http://news.centurylink.com/2018-04-17-Botnets-remain-a-persistent-cyberthreat}
%\href{https://www.virusbulletin.com/blog/2018/09/vb2018-preview-iot-botnets/}
%\href{https://www.cbronline.com/news/kaspersky-botnet-activity}
This is a compilation of Botnet news by Trend Micro: %\href{https://www.trendmicro.com/vinfo/us/security/news/botnets}
Compilation of Botnets variants: %\href{https://www.cyber.nj.gov/threat-profiles/botnet-variants/}
%\href{https://www.theregister.co.uk/2018/06/01/mirai_respun_in_new_botnets/}
\section{Research question}
My research question is : "\textbf{How can we detect Botnets through passive traffic analysis?}"
TODO: define proper sub-questions that build the paper.
With the following sub-questions:
\begin{itemize}
\item What are the current trends of Evasion and detection?
\item Can we create a solution that detects effectively all botnets ?
\item Can I find features not yet exploited to improve detection?
\item Can I create an all-in model fitted for purposes not yet covered by other solutions.
\item What effective model can help organisations detect botnets faster and more reliably?
\end{itemize}
\section{Thesis objective}
The objective of this master thesis in Cyber Security is to improve the all-in solutions for botnet detection through DNS traffic analysis with a machine learning approach and acquire along the way correctly conduct a research process and learn machine learning techniques to solve interesting problems. Different papers haven't always provided a deep study of the choice of features or provided a model adapted to certain environments which we will try to do and see if this can improve the existing models.\\

[source]\\
%[Recognising Botnets in Organisations] Barry Weymes, Master thesis, Department of Computer Science, Radboud University, August 2012
The objective of this thesis is to answer the research questions above. This will be done by reviewing the current (scientific) literature on botnets, in particular with relation to DNS, Honeypots, Law and future threats. The authors technical experience will also be used to answer the research questions. It is hoped that others can benefit from the knowledge that this thesis provides to implement
better security measures within their own organisations. The information provided in this thesis will provide the reader with in depth knowledge on the subject of botnets, and how they the threat of them within an organisation can be migrated.
\\
TODO: Description of the thesis (chapter by chapter)

%----------------------------------------------------------------------------------------
